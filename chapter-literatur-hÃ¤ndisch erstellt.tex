% !TEX root = ../my-thesis.tex
%
\chapter{Literaturverzeichnis}
\label{sec:literatur}
\begin{description}

\item[Badinter, Elisabeth]1991 [1981]: \textit{Die Mutterliebe. Geschichte eines Gefühls vom 17. Jahrhundert bis heute.} München/Zürich: Piper Verlag.

\item[Bandau, Anja]2010 [2001]: Malinche, Malinchismo, Malinchista. Paradigmen für Entwürfe von Chicana-Identität. In: B. Dröscher, C. Rincón (Hg.), \textit{La Malinche. Übersetzung, Interkulturalität und Geschlecht.} Berlin: Verlag Walter Frey, 162-189.

\item[Butler, Judith]1991: \textit{Das Unbehagen der Geschlechter.} Frankfurt a.M.: Suhrkamp
Verlag.

\item[Castro-Urioste, José]1994: La Imagen de Nacion en \glqq Doña Bárbara\grqq{}. \textit{Revista de Crítica Literaria Latinoamericana} 20 (39): 127-139. 

\item[Castro Varela, María do Mar und Paul Mecheril]2016: \textit{Die Dämonisierung der Anderen. Rassismuskritik der Gegenwart.} Bielefeld: transcript Verlag.

\item[Dane, Gesa]2005: \textit{\frqq Zeter und Mordio.\flqq{} Vergewaltigung in Literatur und Recht.} Göttingen: Wallstein Verlag.

\item[Federici, Silvia]2013 [2011]: \textit{Calibán y la Bruja. Mujeres, cuerpo y acumulación originaria.} Distrito Federal de México: Pez en el árbol Ediciones.

\item[Gallegos, Rómulo]1952: \textit{Doña Bárbara.} Zürich: Manesse Verlag.

\item[Gallegos, Rómulo]1954 [1929]: \textit{Doña Bárbara.} Distrito Federal de México: Fondo de Cultura Económica.

\item[Hausen, Karin]2007 [1976]: Die Polarisierung der Geschlechtercharaktere. In: S. Hark (Hg.), \textit{DisKontinuitäten. Feministische Theorie.} Wiesbaden: Springer VS, 173-196. 

\item[Lorey, Isabell]2015: Freiheit und Sorge. Das Recht auf Sorge im Regime der Prekarisierung. In: S. Völker, M. Amacker (Hg.), \textit{Prekarisierungen. Arbeit, Sorge und Politik.} Weinheim/Basel: Beltz Juventa, 26-41.


\end{description}

